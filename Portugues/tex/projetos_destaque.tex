\cvsection{Projetos em Destaque}

\cvevent{Protótipo do Sistema de Cadastro de Demandas do PUC Tec}{
\cvreference{\faGithub}{https://github.com/Henriquemcc/Prototipo_Sistema_Cadastro_Demandas_-_PUC_Tec}
}{09 2024 - 12 2024}{}
Protótipo do Back-End Sistema de Cadastro de Demandas do PUC Tec, desenvolvido no Serviço de Apoio a Startups do PUC Tec (SA PUC Tec) para atender uma demanda interna do PUC Tec.\\
\divider

\cvevent{Protótipo do Sistema da Legião dos Corretores}{
\cvreference{\faGithub}{https://github.com/Henriquemcc/Prototipo_Sistema_Legiao_dos_Corretores}
}{10 2024 - 11 2024}{}
Protótipo do sistema Back-End em Kotlin Spring Boot, desenvolvido no Serviço de Apoio a Startups do PUC Tec (SA PUC Tec) para atender a demanda da startup Legião dos Corretores.\\
\divider

\cvevent{Simulação Detector de Falhas com Consenso - Computação Distribuída}{
\cvreference{\faGithub}{https://github.com/Henriquemcc/Simulacao_Detector_de_Falhas_com_Consenso_-_Computacao_Distribuida_-_2024-2}
}{11 2024}{}
Trabalho da matéria Computação Distribuída no qual deveria ser desenvolvido uma simulação do funcionamento do detector de falhas com consenso.\\
\divider

\cvevent{Simulação - Exclusão Mútua Distribuída - Computação Distribuída}{
\cvreference{\faGithub}{https://github.com/Henriquemcc/Simulacao_-_Exclusao_Mutua_Distribuida_-_Computacao_Distribuida_-_2024-2}
}{10 2024}{}
Trabalho da matéria Computação Distribuída no qual deveria ser desenvolvido uma simulação do funcionamento de algoritmos de Exclusão Mútua Distribuída.\\
\divider

\cvevent{Trabalho Prático - Projeto e Análise de Algoritmos}{
\cvreference{\faGithub}{https://github.com/Henriquemcc/Trabalho_Pratico_-_Projeto_e_Analise_de_Algoritmos_-_2024-1}
}{04 2024 - 06 2024}{}
Trabalho prático da disciplina Projeto e Análise de Algoritmos no qual deveria ser desenvolvido um programa com interface gráfica que resolvesse problemas de otimização da localização de instalações industriais ou comerciais em uma determinada região de interesse.\\
\divider

\cvevent{Laboratório de Criação de Servidor Web no Microsoft Azure}{
\cvreference{\faGithub}{https://github.com/WebTech-PUC-Minas/lab-azure-web-server}
}{04 2024 - 04 2024}{}
Trabalho de extensão desenvolvido no WebTech PUC Minas no qual foi desenvolvido um laboratório de como criar um simples servidor web no Microsoft Azure.\\
\divider

\cvevent{Paint}{
\cvreference{\faGithub}{https://github.com/Henriquemcc/Trabalho_Pratico_-_Computacao_Grafica_-_2024-1}
}{02 2024 -- 03 2024}{}
Trabalho prático da disciplina Computação Gráfica no qual deveria ser desenvolvido um programa com interface gráfica que utilizasse algoritmos de computação gráfica para a inserção de retas, pontos, circunferência e polígonos, além da realização de operações gráficas de rotação, translação, reflexão e escala.\\
\divider

\cvevent{Chat em Java}
{
\cvreference{\faGithub}{https://github.com/Henriquemcc/Lista_5_-_Unidade_III_-_Chat_em_Java_-_Redes_de_Computadores_I}
}{08 2023 -- 09 2023}{}
Trabalho em grupo desenvolvido na matéria Redes de Computadores I, no qual uma aplicação cliente e servidor deveria ser desenvolvida em Java, utilizando os protocolos (da camada de transporte) TCP e UDP. Para este trabalho, foi desenvolvido um programa de mensagens (chat) no qual dois ou mais clientes, através de um servidor, conseguem trocar mensagens entre si.\\
\divider

{\large\color{emphasis}\href{https://www.linkedin.com/in/henriquemcc/details/projects/}{Mais informações no LinkedIn}}
\cvsection{Projetos em Destaque}

\cvevent{AS05: Implementação de Assistente Conversacional Baseado em LLM - Tópicos em Computação III}{
\cvreference{\faGithub}{https://github.com/Henriquemcc/AS05_Implementacao_de_Assistente_Conversacional_Baseado_em_LLM_-_Topicos_em_Computacao_III_-_2025-1}
}{06 2025 - 07 2025}{}
Trabalho da matéria Tópicos em Computação III do curso de Ciência da Computação da PUC Minas no qual deveria ser desenvolvido um Assistente Conversacional Baseado em LLM que seja capaz de indexar vetores (embeddings textuais) de uma coleção de documentos em PDF para responder posteriormente a perguntas feitas por meio de uma interface de conversação.\\
\divider

\cvevent{Bug Watch}{\cvreference{\faGitlab}{https://gitlab.com/henriquemcc/bug_watch}}{08 2024 - 11 2024}{}
Trabalho em grupo da disciplina Trabalho Interdisciplinar VI no qual deveria ser desenvolvido uma aplicação distribuída, paralela e que utilizasse visão computacional. Neste trabalho, foi desenvolvido uma aplicação web de classificação de insetos através de imagens.\\
\divider

\cvevent{Trabalho Prático - Processamento e Análise de Imagens}{
\cvreference{\faGithub}{https://github.com/Henriquemcc/Trabalho_Pratico_-_Processamento_e_Analise_de_Imagens_-_2024-1}
}{03 2024 - 06 2024}{}
Trabalho prático da disciplina Processamento e Análise de Imagens no qual deveria ser desenvolvido um programa com interface gráfica que realizasse o processamento e a classificação, com o uso de técnicas de aprendizado de máquina, de imagens de células de exame médico.\\
\divider

{\large\color{emphasis}\href{https://www.linkedin.com/in/henriquemcc/details/projects/}{Mais informações no LinkedIn}}
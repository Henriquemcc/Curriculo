\cvsection{Projetos Desenvolvidos na faculdade}

\cvevent{AS05: Implementação de Assistente Conversacional Baseado em LLM - Tópicos em Computação III}{
\cvreference{\faGithub}{https://github.com/Henriquemcc/AS05_Implementacao_de_Assistente_Conversacional_Baseado_em_LLM_-_Topicos_em_Computacao_III_-_2025-1}
}{06 2025 - 07 2025}{}
Trabalho da matéria Tópicos em Computação III do curso de Ciência da Computação da PUC Minas no qual deveria ser desenvolvido um Assistente Conversacional Baseado em LLM que seja capaz de indexar vetores (embeddings textuais) de uma coleção de documentos em PDF para responder posteriormente a perguntas feitas por meio de uma interface de conversação.
\divider

\cvevent{Robot Exterminator - Trabalho Interdisciplinar IV - 2025-1}{
\cvreference{\faGitlab}{https://gitlab.com/henriquemcc/robot-exterminator-trabalho-interdisciplinar-iv-2025-1}
}{03 2025 - 06 2025}{}
Trabalho prático da disciplina Trabalho Interdisciplinar IV do curso de Ciência da Computação da Pontifícia Universidade Católica de Minas Gerais.
\divider

\cvevent{Trabalho Prático 1 - Compiladores}{
\cvreference{\faGithub}{https://github.com/Henriquemcc/Trabalho_Pratico_1_-_Compiladores_-_2024-2}
}{10 2024 - 12 2024}{}
Trabalho prático da disciplina Compiladores do curso de Ciência da Computação da Pontifícia Universidade Católica de Minas Gerais.\\
\divider

\cvevent{Bug Watch}{\cvreference{\faGitlab}{https://gitlab.com/henriquemcc/bug_watch}}{08 2024 - 11 2024}{}
Trabalho em grupo da disciplina Trabalho Interdisciplinar VI no qual deveria ser desenvolvido uma aplicação distribuída, paralela e que utilizasse visão computacional. Neste trabalho, foi desenvolvido uma aplicação web de classificação de insetos através de imagens.\\
\divider

\cvevent{Simulação Detector de Falhas com Consenso - Computação Distribuída}{
\cvreference{\faGithub}{https://github.com/Henriquemcc/Simulacao_Detector_de_Falhas_com_Consenso_-_Computacao_Distribuida_-_2024-2}
}{11 2024}{}
Trabalho da matéria Computação Distribuída no qual deveria ser desenvolvido uma simulação do funcionamento do detector de falhas com consenso.\\
\divider

\cvevent{Simulação - Exclusão Mútua Distribuída - Computação Distribuída}{
\cvreference{\faGithub}{https://github.com/Henriquemcc/Simulacao_-_Exclusao_Mutua_Distribuida_-_Computacao_Distribuida_-_2024-2}
}{10 2024}{}
Trabalho da matéria Computação Distribuída no qual deveria ser desenvolvido uma simulação do funcionamento de algoritmos de Exclusão Mútua Distribuída.\\
\divider

\cvevent{Trabalho Prático - Processamento e Análise de Imagens}{
\cvreference{\faGithub}{https://github.com/Henriquemcc/Trabalho_Pratico_-_Processamento_e_Analise_de_Imagens_-_2024-1}
}{03 2024 - 06 2024}{}
Trabalho prático da disciplina Processamento e Análise de Imagens no qual deveria ser desenvolvido um programa com interface gráfica que realizasse o processamento e a classificação, com o uso de técnicas de aprendizado de máquina, de imagens de células de exame médico.\\
\divider

\cvevent{Trabalho Prático - Projeto e Análise de Algoritmos}{
\cvreference{\faGithub}{https://github.com/Henriquemcc/Trabalho_Pratico_-_Projeto_e_Analise_de_Algoritmos_-_2024-1}
}{04 2024 - 06 2024}{}
Trabalho prático da disciplina Projeto e Análise de Algoritmos no qual deveria ser desenvolvido um programa com interface gráfica que resolvesse problemas de otimização da localização de instalações industriais ou comerciais em uma determinada região de interesse.\\
\divider

\cvevent{Paint}{
\cvreference{\faGithub}{https://github.com/Henriquemcc/Trabalho_Pratico_-_Computacao_Grafica_-_2024-1}
}{02 2024 -- 03 2024}{}
Trabalho prático da disciplina Computação Gráfica no qual deveria ser desenvolvido um programa com interface gráfica que utilizasse algoritmos de computação gráfica para a inserção de retas, pontos, circunferência e polígonos, além da realização de operações gráficas de rotação, translação, reflexão e escala.\\
\divider

\cvevent{Implementação Paralela de Blockchain}
{
\cvreference{\faGithub}{https://github.com/Henriquemcc/Projeto_02_-_Implementacao_Paralela_de_Blockchain_-_Computacao_Paralela_-_2023-2}
}{10 2023 -- 12 2023}{}
Trabalho em grupo desenvolvido na matéria Computação Paralela, no qual um código de Blockchain deveria ser paralelizado utilizando Open MP para Multicore, Open MP para GPU e Nvidia CUDA.\\
\divider

\cvevent{Chat em Java}
{
\cvreference{\faGithub}{https://github.com/Henriquemcc/Lista_5_-_Unidade_III_-_Chat_em_Java_-_Redes_de_Computadores_I}
}{08 2023 -- 09 2023}{}
Trabalho em grupo desenvolvido na matéria Redes de Computadores I, no qual uma aplicação cliente e servidor deveria ser desenvolvida em Java, utilizando os protocolos (da camada de transporte) TCP e UDP. Para este trabalho, foi desenvolvido um programa de mensagens (chat) no qual dois ou mais clientes, através de um servidor, conseguem trocar mensagens entre si.\\
\divider
        
\cvevent{Jogo 8-Puzzle}
{
\cvreference{\faGithub}{https://github.com/Henriquemcc/Jogo_-_Inteligencia_Artificial_2023-1}
\cvreference{|\faGlobe}{https://henriquemcc.github.io/Jogo_-_Inteligencia_Artificial_2023-1/}}{02 2023 -- 06 2023}{}
Trabalho desenvolvido na matéria Inteligência Artificial, no qual métodos de busca em grafos (A Estrela, Busca Gulosa e Busca Uniforme) deveriam ser utilizados para encontrar uma solução para o Jogo 8-Puzzle.\\
\divider

\cvevent{Portal de Filmes}
{
\cvreference{\faGithub}{https://github.com/Henriquemcc/Portal_de_Filmes_-_Desenvolvimento_de_Interfaces_Web}
\cvreference{|\faGlobe}{https://henriquemcc.github.io/Portal_de_Filmes_-_Desenvolvimento_de_Interfaces_Web/}}{04 2022 -- 06 2022}{}
Trabalho prático da matéria Desenvolvimento de Interfaces Web da Pontifícia Universidade Católica de Minas Gerais\\
\divider

{\large\color{emphasis}\href{https://www.linkedin.com/in/henriquemcc/details/projects/}{Mais informações no LinkedIn}}
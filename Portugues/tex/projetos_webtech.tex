\cvsection{Projetos Desenvolvidos no WebTech}

\cvevent{DockerLab: Criando e Gerenciando Containers de forma prática}{
\cvreference{\faGlobe}{https://www.sympla.com.br/evento/dockerlab-criando-e-gerenciando-containers-de-forma-pratica/2950892}
}{05 2025 - 05 2025}{}
Workshop de Docker: dos comandos básicos até como containerizar uma aplicação e colocá-la na nuvem do Microsoft Azure.\\
\divider

\cvevent{Laboratório de Docker}{
\cvreference{\faGithub}{https://github.com/webtech-network/lab-docker}
}{02 2025 - 05 2025}{}
Laboratório de Docker para quem nunca mexeu com Docker.\\
\divider

\cvevent{Workshop WebTech - Linux para iniciantes}{
\cvreference{\faGlobe}{https://www.sympla.com.br/evento/workshop-webtech-linux-para-iniciantes/2904736}
}{04 2024 - 04 2024}{}
Workshop de como instalar, configurar e utilizar distribuições Linux para quem nunca utilizou Linux.\\
\divider

\cvevent{Laboratório de Linux para Iniciantes}{
\cvreference{\faGithub}{https://github.com/webtech-network/lab-linux-iniciantes}
}{08 2024 - 12 2024}{}
Laboratório de como instalar, configurar e utilizar distribuições Linux para quem nunca utilizou Linux.\\
\divider

\cvevent{WebTech Workshop - DevOps - CI/CD com GitHub Actions}{
\cvreference{\faGlobe}{https://www.sympla.com.br/evento/webtech-workshop-devops-ci-cd-com-github-actions/2478558}
}{06 2024 - 06 2024}{}
Workshop de como elaborar uma workflow no GitHub Actions para integrar automaticamente um repositório do GitHub com um servidor web no Microsoft Azure.\\
\divider
        
\cvevent{Laboratório de Criação de Servidor Web no Microsoft Azure}{
\cvreference{\faGithub}{https://github.com/WebTech-PUC-Minas/lab-azure-web-server}
}{04 2024 - 04 2024}{}
Trabalho de extensão desenvolvido no WebTech PUC Minas no qual foi desenvolvido um laboratório de como criar um simples servidor web no Microsoft Azure.\\
\divider
        
\cvevent{Laboratório de Deploy automático com GitHub Actions}{
\cvreference{\faGithub}{https://github.com/WebTech-PUC-Minas/lab-devops-github-actions}
}{03 2024 -- 04 2024}{}
Trabalho de extensão desenvolvido no WebTech PUC Minas no qual foi desenvolvido um laboratório de como elaborar uma workflow no GitHub Actions para integrar automaticamente um repositório do GitHub com um servidor web.\\
\divider

{\large\color{emphasis}\href{https://www.linkedin.com/in/henriquemcc/details/projects/}{Mais informações no LinkedIn}}
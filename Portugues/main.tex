%%%%%%%%%%%%%%%%%
% This is an sample CV template created using altacv.cls
% (v1.3, 10 May 2020) written by LianTze Lim (liantze@gmail.com). Now compiles with pdfLaTeX, XeLaTeX and LuaLaTeX.
% (v1.6.5b, 27 Jun 2023) forked by Nicolás Omar González Passerino (nicolas.passerino@gmail.com)
%
%% It may be distributed and/or modified under the
%% conditions of the LaTeX Project Public License, either version 1.3
%% of this license or (at your option) any later version.
%% The latest version of this license is in
%%    http://www.latex-project.org/lppl.txt
%% and version 1.3 or later is part of all distributions of LaTeX
%% version 2003/12/01 or later.
%%%%%%%%%%%%%%%%

%% If you need to pass whatever options to xcolor
\PassOptionsToPackage{dvipsnames}{xcolor}

%% If you are using \orcid or academicons
%% icons, make sure you have the academicons
%% option here, and compile with XeLaTeX
%% or LuaLaTeX.
% \documentclass[10pt,a4paper,academicons]{altacv}

%% Use the "normalphoto" option if you want a normal photo instead of cropped to a circle
% \documentclass[10pt,a4paper,normalphoto]{altacv}

%% Fork (before v1.6.5a): CV dark mode toggle enabler to use a inverted color palette.
%% Use the "darkmode" option if you want a color palette used to 
% \documentclass[10pt,a4paper,ragged2e,withhyper,darkmode]{altacv}

\documentclass[10pt,a4paper,ragged2e,withhyper]{altacv}

%% AltaCV uses the fontawesome5 and academicons fonts
%% and packages.
%% See http://texdoc.net/pkg/fontawesome5 and http://texdoc.net/pkg/academicons for full list of symbols. You MUST compile with XeLaTeX or LuaLaTeX if you want to use academicons.

% Change the page layout if you need to
\geometry{left=1.2cm,right=1.2cm,top=1cm,bottom=1cm,columnsep=0.75cm}

% The paracol package lets you typeset columns of text in parallel
\usepackage{paracol}
\usepackage{hyperref}

% Change the font if you want to, depending on whether
% you're using pdflatex or xelatex/lualatex
\ifxetexorluatex
  % If using xelatex or lualatex:
  \setmainfont{Roboto Slab}
  \setsansfont{Lato}
  \renewcommand{\familydefault}{\sfdefault}
\else
  % If using pdflatex:
  \usepackage[rm]{roboto}
  \usepackage[defaultsans]{lato}
  % \usepackage{sourcesanspro}
  \renewcommand{\familydefault}{\sfdefault}
\fi

% Fork (before v1.6.5a): Change the color codes to test your personal variant on any mode
\ifdarkmode%
  \definecolor{PrimaryColor}{HTML}{C69749}
  \definecolor{SecondaryColor}{HTML}{D49B54}
  \definecolor{ThirdColor}{HTML}{1877E8}
  \definecolor{BodyColor}{HTML}{ABABAB}
  \definecolor{EmphasisColor}{HTML}{ABABAB}
  \definecolor{BackgroundColor}{HTML}{191919}
\else%
  \definecolor{PrimaryColor}{HTML}{001F5A}
  \definecolor{SecondaryColor}{HTML}{0039AC}
  \definecolor{ThirdColor}{HTML}{F3890B}
  \definecolor{BodyColor}{HTML}{666666}
  \definecolor{EmphasisColor}{HTML}{2E2E2E}
  \definecolor{BackgroundColor}{HTML}{E2E2E2}
\fi%

\colorlet{name}{PrimaryColor}
\colorlet{tagline}{SecondaryColor}
\colorlet{heading}{PrimaryColor}
\colorlet{headingrule}{ThirdColor}
\colorlet{subheading}{SecondaryColor}
\colorlet{accent}{SecondaryColor}
\colorlet{emphasis}{EmphasisColor}
\colorlet{body}{BodyColor}
\pagecolor{BackgroundColor}

% Change some fonts, if necessary
\renewcommand{\namefont}{\Huge\rmfamily\bfseries}
\renewcommand{\personalinfofont}{\small\bfseries}
\renewcommand{\cvsectionfont}{\LARGE\rmfamily\bfseries}
\renewcommand{\cvsubsectionfont}{\large\bfseries}

% Change the bullets for itemize and rating marker
% for \cvskill if you want to
\renewcommand{\itemmarker}{{\small\textbullet}}
\renewcommand{\ratingmarker}{\faCircle}

%% sample.bib contains your publications
%% \addbibresource{main.bib}

\begin{document}
    \name{Henrique Mendonça Castelar Campos}
    \tagline{Estudante de Ciência da Computação}
    %% You can add multiple photos on the left or right
    \photoL{4cm}{1702063880079}

    \personalinfo{
        \email{henriquemendonacastelar@gmail.com}\smallskip
        \phone{+55 31 9 9506-6143}
        \location{Belo Horizonte, Minas Gerais, Brasil}\\
        \linkedin{henriquemcc}
        \github{Henriquemcc}
        \dockerhub{Henriquemcc}
        \credly{henriquemcc}
        % \npm{}
        % \dev{henriquemcc}
        \homepage{henriquemcc.github.io}
        %\medium{}
        %% You MUST add the academicons option to \documentclass, then compile with LuaLaTeX or XeLaTeX, if you want to use \orcid or other academicons commands.
        % \orcid{0000-0002-8726-8936}
        %% You can add your own arbtrary detail with
        %% \printinfo{symbol}{detail}[optional hyperlink prefix]
        % \printinfo{\faPaw}{Hey ho!}[https://example.com/]
        %% Or you can declare your own field with
        %% \NewInfoFiled{fieldname}{symbol}[optional hyperlink prefix] and use it:
        % \NewInfoField{gitlab}{\faGitlab}[https://gitlab.com/]
        \gitlab{henriquemcc}
    }
    
    \makecvheader
    %% Depending on your tastes, you may want to make fonts of itemize environments slightly smaller
    % \AtBeginEnvironment{itemize}{\small}
    
    %% Set the left/right column width ratio to 6:4.
    \columnratio{0.25}

    % Start a 2-column paracol. Both the left and right columns will automatically
    % break across pages if things get too long.
    \begin{paracol}{2}
        % ----- Competências -----
        \cvsection{Tecnologias que domino}
            \cvtag{Spring Boot}
            \cvtag{Kotlin}
            \cvtag{Python}
            \cvtag{Git}
            \cvtag{GitHub}
            \cvtag{AWS}
            \cvtag{Google Cloud}
            \cvtag{Shell Scripting}
            \cvtag{Linux}
            \cvtag{Java}
            \cvtag{Docker}
            
          
        \divider
        
        \cvsection{Tecnologias que estou familiarizado}
            \cvtag{Azure}
            \cvtag{PowerShell}\\
            \cvtag{JavaScript}
            \cvtag{HTML5}
            \cvtag{CSS}
            \cvtag{Go}
            \cvtag{C++}
            \cvtag{C\#}
            \cvtag{C}
            \cvtag{\LaTeX}
            \cvtag{Android}\\
            \cvtag{Open MP}
            \cvtag{Nvidia Cuda}
            \cvtag{Node.js}

        \divider
            
            {\large\color{emphasis}\href{https://www.linkedin.com/in/henriquemcc/details/skills/}{Mais informações no LinkedIn}}
        % ----- Competências -----
        
        % ----- LEARNING -----
        % \cvsection{Learning}
        %     \cvtag{Uno}
        %     \cvtag{Dos}
        %     \cvtag{Tres}
        %     \cvtag{Cuatro}
        %     \cvtag{Cinco}
        %     \cvtag{Seis}
        %     \cvtag{Siete}
        %     \cvtag{Ocho}
        %     \cvtag{Nueve}
        %     \cvtag{Diez}
        %     \medskip
            
        %     \cvtag{Rojo}
        %     \cvtag{Amarillo}
        %     \cvtag{Azul}
        %     \cvtag{Verde}
        %     \cvtag{Violeta}
        %     \cvtag{Naranja}
        %     \cvtag{Marron}
        %     \cvtag{Blanco}
        %     \cvtag{Gris}
        %     \cvtag{Negro}
        % ----- LEARNING -----
        
        % ----- LANGUAGES -----
        % \cvsection{Languages}
        %     \cvlang{Lang 1}{Native}\\
        %     \divider

        %     \cvlang{Lang 2}{Basic / A2}
            %% Yeah I didn't spend too much time making all the
            %% spacing consistent... sorry. Use \smallskip, \medskip,
            %% \bigskip, \vpsace etc to make ajustments.
        % ----- LANGUAGES -----
            
        % ----- REFERENCES -----
        % \cvsection{References}
        %     \cvref{Prof.\ Alpha Beta}{Institute}{a.beta@university.edu}
        %     \divider

        %     \cvref{Boss\ Gamma Delta}{Business}{g.delta@business.com}
        % ----- REFERENCES -----
        
        % ----- MOST PROUD -----
        % \cvsection{Most Proud of}
        
        % \cvachievement{\faTrophy}{Fantastic Achievement}{and some details about it}\\
        % \divider
        % \cvachievement{\faHeartbeat}{Another achievement}{more details about it of course}\\
        % \divider
        % \cvachievement{\faHeartbeat}{Another achievement}{more details about it of course}
        % ----- MOST PROUD -----
        
        % \cvsection{A Day of My Life}
        
        % Adapted from @Jake's answer from http://tex.stackexchange.com/a/82729/226
        % \wheelchart{outer radius}{inner radius}{
        % comma-separated list of value/text width/color/detail}
        % \wheelchart{1.5cm}{0.5cm}{%
        %   6/8em/accent!30/{Sleep,\\beautiful sleep},
        %   3/8em/accent!40/Hopeful novelist by night,
        %   8/8em/accent!60/Daytime job,
        %   2/10em/accent/Sports and relaxation,
        %   5/6em/accent!20/Spending time with family
        % }
        
        % use ONLY \newpage if you want to force a page break for
        % ONLY the current column
        \newpage
        
        %% Switch to the right column. This will now automatically move to the second
        %% page if the content is too long.
        \switchcolumn
        
        % ----- ABOUT ME -----
        \cvsection{Cursos e Escolaridade}
            \begin{quote}
                Estudo Ciência da Computação na Pontifícia Universidade Católica de Minas Gerais, atualmente estou cursando o último período, com previsão de formar em agosto de 2025. Interessado em Desenvolvimento Back-End, DevOps, e Computação em Nuvem.
            \end{quote}
        % ----- ABOUT ME -----
        
        % ----- Experiência -----
        \cvsection{Experiência}
            \cvevent{Desenvolvedor full stack}{WebTech Network}{02 2024 --}{Belo Horizonte, Minas Gerais, Brasil}
            Projeto de extensão da PUC Minas para gerar espaço de colaboração onde alunos, professores, empresas e interessados aprendam juntos e gerem conteúdo, aplicações e ferramentas para soluções digitais.

            \divider
            \cvevent{Full Stack Developer}{PUCTec}{08 2024 -- 12 2024}{Belo Horizonte, Minas Gerais, Brasil}
            Desenvolvimento de software sobre demanda das startups atendidas pelo Serviço de Apoio à Startups do PUC Tec (SA PUC Tec).
            
            
            \divider
            \cvevent{Monitor de Ciência da Computação}{Pontifícia Universidade Católica de Minas Gerais}{08 2023 -- 12 2023}{Belo Horizonte, Minas Gerais, Brasil}
            Monitoria da disciplina Teoria dos Grafos e Computabilidade. Durante a monitoria, minha função consistia em auxiliar os alunos na resolução de dúvidas relacionadas a disciplina.

            \divider
            \cvevent{Estagiário}{Prefeitura de Belo Horizonte}{09 2022 -- 11 2022}{Belo Horizonte, Minas Gerais, Brasil}
            Atendimento interno e externo, tipo help Desk, no sistema SGPREV.
            
            
            % \divider
        % ----- Experiência -----
        
        % ----- Formação acadêmica -----
        \cvsection{Formação acadêmica}
            \cvevent{Bacharelado, Ciência da Computação}{Pontifícia Universidade Católica de Minas Gerais
}{02 2018 -- 08 2025}{Belo Horizonte, Minas Gerais, Brasil}
            % \divider
        % ----- Formação acadêmica -----

        % ----- Certificações -----
        \cvsection{Certificações}

        \cvevent{AWS Certified Cloud Practitioner}{\cvreference{\faGlobe}{https://www.credly.com/badges/7623bc5f-4a7a-49d9-9504-26b399105745/linked_in_profile}}{01 2024}{Belo Horizonte, Minas Gerais, Brasil}
        \divider

        \cvevent{B2 First – Score 172 - Cambridge University Press \& Assessment English}{\cvreference{\faGlobe}{https://drive.google.com/file/d/1XlpfYXp5Veeiyn8zAHABk8SSAO36QncZ/view?usp=sharing}}{09 2022}{Belo Horizonte, Minas Gerais, Brasil}
        \divider

        {\large\color{emphasis} \href{https://www.linkedin.com/in/henriquemcc/details/certifications/}{Mais informações no LinkedIn}}

        % ----- Certificados -----
        \cvsection{Certificados}

        \cvevent{Google Cloud Computing Foundations Certificate}{\cvreference{\faGlobe}{https://www.credly.com/badges/743d3065-9d81-43e4-ae9b-73858bdb30e2/linked_in_profile}}
        {05 2025}{Online}
        \divider

        \cvevent{Formação Kotlin e Spring Boot - Alura}{\cvreference{\faGlobe}{https://cursos.alura.com.br/degree/certificate/09cbf228-4894-4225-bf0e-6be933822ffe?lang}\cvreference{\faGithub}{https://github.com/Henriquemcc/Forum_-_Formacao_Kotlin_e_Spring_Boot_-_Alura}}{03 2025}{Online}
        \divider
        
        \cvevent{React: comece seu projeto full stack - Alura}{\cvreference{\faGlobe}{https://cursos.alura.com.br/certificate/b4f48602-24d3-4fce-928f-0038f178004a?lang}\cvreference{\faGithub}{https://github.com/Henriquemcc/Alura_Books_-_Formacao_Full_stack_JavaScript_-_Alura}}{01 2025}{Online}
        \divider
        
        \cvevent{Endpoint Security - Cisco}{\cvreference{\faGlobe}{https://www.credly.com/badges/dd945a87-ba32-4732-9ea5-88198208599f/linked_in_profile}}{03 2024}{Online}
        \divider
        
        \cvevent{Cursos da Alura}{\cvreference{\faGlobe}{https://cursos.alura.com.br/user/henriquemcc/fullCertificate/ebc4dcd6245bdf46e4d6ffd89a1e3ec2}}{07 2020 - 12 2023}{Online}
        \divider
        
        \cvevent{Networking Basics - Cisco}{\cvreference{\faGlobe}{https://www.credly.com/badges/c9830260-5298-434e-8955-4eb876480ba6/linked_in_profile}}{09 2022}{Online}
        \divider
        
        \cvevent{Introduction to Cybersecurity - Cisco}{\cvreference{\faGlobe}{https://www.credly.com/badges/4676e79d-3e11-4856-afc7-38b96e1edc95/linked_in_profile}}{08 2022}{Online}
        \divider
        
        \cvevent{Formação Linguagem Kotlin - Alura}{\cvreference{\faGlobe}{https://cursos.alura.com.br/degree/certificate/18f608ec-a511-43b4-8586-04c87b079a4c?lang}\cvreference{\faGithub}{https://github.com/Henriquemcc/Aprendendo_Kotlin}}{07 2021}{Online}
        \divider
        
        \cvevent{Formação Python - Alura}{\cvreference{\faGlobe}{https://cursos.alura.com.br/degree/certificate/b96bda48-dc02-4105-9ca5-ae64c2e135e3?lang}}{08 2020}{Online}
        \divider
        
        
        {\large\color{emphasis} \href{https://www.linkedin.com/in/henriquemcc/details/certifications/}{Mais informações no LinkedIn}}
        
        % ----- Certificados -----
        
        % ----- Projetos -----
        \cvsection{Projetos Desenvolvidos no WebTech}

            \cvevent{DockerLab: Criando e Gerenciando Containers de forma prática}{
            \cvreference{\faGlobe}{https://www.sympla.com.br/evento/dockerlab-criando-e-gerenciando-containers-de-forma-pratica/2950892}
            }{05 2025 - 05 2025}{}
            Workshop de Docker: dos comandos básicos até como containerizar uma aplicação e colocá-la na nuvem do Microsoft Azure.\\
            \divider

            \cvevent{Laboratório de Docker}{
            \cvreference{\faGithub}{https://github.com/webtech-network/lab-docker}
            }{02 2025 - 05 2025}{}
            Laboratório de Docker para quem nunca mexeu com Docker.\\
            \divider

            \cvevent{Workshop WebTech - Linux para iniciantes}{
            \cvreference{\faGlobe}{https://www.sympla.com.br/evento/workshop-webtech-linux-para-iniciantes/2904736}
            }{04 2024 - 04 2024}{}
            Workshop de como instalar, configurar e utilizar distribuições Linux para quem nunca utilizou Linux.\\
            \divider

            \cvevent{Laboratório de Linux para Iniciantes}{
            \cvreference{\faGithub}{https://github.com/webtech-network/lab-linux-iniciantes}
            }{08 2024 - 12 2024}{}
            Laboratório de como instalar, configurar e utilizar distribuições Linux para quem nunca utilizou Linux.\\
            \divider

            \cvevent{WebTech Workshop - DevOps - CI/CD com GitHub Actions}{
            \cvreference{\faGlobe}{https://www.sympla.com.br/evento/webtech-workshop-devops-ci-cd-com-github-actions/2478558}
            }{06 2024 - 06 2024}{}
            Workshop de como elaborar uma workflow no GitHub Actions para integrar automaticamente um repositório do GitHub com um servidor web no Microsoft Azure.\\
            \divider
        
            \cvevent{Laboratório de Criação de Servidor Web no Microsoft Azure}{
            \cvreference{\faGithub}{https://github.com/WebTech-PUC-Minas/lab-azure-web-server}
            }{04 2024 - 04 2024}{}
            Trabalho de extensão desenvolvido no WebTech PUC Minas no qual foi desenvolvido um laboratório de como criar um simples servidor web no Microsoft Azure.\\
            \divider
        
            \cvevent{Laboratório de Deploy automático com GitHub Actions}{
            \cvreference{\faGithub}{https://github.com/WebTech-PUC-Minas/lab-devops-github-actions}
            }{03 2024 -- 04 2024}{}
            Trabalho de extensão desenvolvido no WebTech PUC Minas no qual foi desenvolvido um laboratório de como elaborar uma workflow no GitHub Actions para integrar automaticamente um repositório do GitHub com um servidor web.\\
            \divider

            {\large\color{emphasis}\href{https://www.linkedin.com/in/henriquemcc/details/projects/}{Mais informações no LinkedIn}}
            
        % ----- Projetos -----

        \cvsection{Projetos Desenvolvidos no PUC Tec}

            \cvevent{Protótipo do Sistema de Cadastro de Demandas do PUC Tec}{
            \cvreference{\faGithub}{https://github.com/Henriquemcc/Prototipo_Sistema_Cadastro_Demandas_-_PUC_Tec}
            }{09 2024 - 12 2024}{}
            Protótipo do Back-End Sistema de Cadastro de Demandas do PUC Tec, desenvolvido no Serviço de Apoio a Startups do PUC Tec (SA PUC Tec) para atender uma demanda interna do PUC Tec.\\
            \divider

            \cvevent{Protótipo do Sistema da Legião dos Corretores}{
            \cvreference{\faGithub}{https://github.com/Henriquemcc/Prototipo_Sistema_Legiao_dos_Corretores}
            }{10 2024 - 11 2024}{}
            Protótipo do sistema Back-End em Kotlin Spring Boot, desenvolvido no Serviço de Apoio a Startups do PUC Tec (SA PUC Tec) para atender a demanda da startup Legião dos Corretores.\\
            \divider

        {\large\color{emphasis}\href{https://www.linkedin.com/in/henriquemcc/details/projects/}{Mais informações no LinkedIn}}

        \cvsection{Projetos Desenvolvidos na faculdade}

            \cvevent{Trabalho Prático 1 - Compiladores}{
            \cvreference{\faGithub}{https://github.com/Henriquemcc/Trabalho_Pratico_1_-_Compiladores_-_2024-2}
            }{10 2024 - 12 2024}{}
            Trabalho prático da disciplina Compiladores do curso de Ciência da Computação da Pontifícia Universidade Católica de Minas Gerais.\\
            \divider

            \cvevent{Bug Watch}{\cvreference{\faGlobe}{https://gitlab.com/henriquemcc/bug_watch}}{08 2024 - 11 2024}{}
            Trabalho em grupo da disciplina Trabalho Interdisciplinar VI no qual deveria ser desenvolvido uma aplicação distribuída, paralela e que utilizasse visão computacional. Neste trabalho, foi desenvolvido uma aplicação web de classificação de insetos através de imagens.\\
            \divider

            \cvevent{Simulação Detector de Falhas com Consenso - Computação Distribuída}{
            \cvreference{\faGithub}{https://github.com/Henriquemcc/Simulacao_Detector_de_Falhas_com_Consenso_-_Computacao_Distribuida_-_2024-2}
            }{11 2024}{}
            Trabalho da matéria Computação Distribuída no qual deveria ser desenvolvido uma simulação do funcionamento do detector de falhas com consenso.\\
            \divider

            \cvevent{Simulação - Exclusão Mútua Distribuída - Computação Distribuída}{
            \cvreference{\faGithub}{https://github.com/Henriquemcc/Simulacao_-_Exclusao_Mutua_Distribuida_-_Computacao_Distribuida_-_2024-2}
            }{10 2024}{}
            Trabalho da matéria Computação Distribuída no qual deveria ser desenvolvido uma simulação do funcionamento de algoritmos de Exclusão Mútua Distribuída.\\
            \divider

            \cvevent{Trabalho Prático - Processamento e Análise de Imagens}{
            \cvreference{\faGithub}{https://github.com/Henriquemcc/Trabalho_Pratico_-_Processamento_e_Analise_de_Imagens_-_2024-1}
            }{03 2024 - 06 2024}{}
            Trabalho prático da disciplina Processamento e Análise de Imagens no qual deveria ser desenvolvido um programa com interface gráfica que realizasse o processamento e a classificação, com o uso de técnicas de aprendizado de máquina, de imagens de células de exame médico.\\
            \divider
            
            \cvevent{Trabalho Prático - Projeto e Análise de Algoritmos}{
            \cvreference{\faGithub}{https://github.com/Henriquemcc/Trabalho_Pratico_-_Projeto_e_Analise_de_Algoritmos_-_2024-1}
            }{04 2024 - 06 2024}{}
            Trabalho prático da disciplina Projeto e Análise de Algoritmos no qual deveria ser desenvolvido um programa com interface gráfica que resolvesse problemas de otimização da localização de instalações industriais ou comerciais em uma determinada região de interesse.\\
            \divider

            \cvevent{Paint}{
            \cvreference{\faGithub}{https://github.com/Henriquemcc/Trabalho_Pratico_-_Computacao_Grafica_-_2024-1}
            }{02 2024 -- 03 2024}{}
            Trabalho prático da disciplina Computação Gráfica no qual deveria ser desenvolvido um programa com interface gráfica que utilizasse algoritmos de computação gráfica para a inserção de retas, pontos, circunferência e polígonos, além da realização de operações gráficas de rotação, translação, reflexão e escala.\\
            \divider

            \cvevent{Implementação Paralela de Blockchain}
            {
            \cvreference{\faGithub}{https://github.com/Henriquemcc/Projeto_02_-_Implementacao_Paralela_de_Blockchain_-_Computacao_Paralela_-_2023-2}
            }{10 2023 -- 12 2023}{}
            Trabalho em grupo desenvolvido na matéria Computação Paralela, no qual um código de Blockchain deveria ser paralelizado utilizando Open MP para Multicore, Open MP para GPU e Nvidia CUDA.\\
            \divider

            \cvevent{Chat em Java}
            {
            \cvreference{\faGithub}{https://github.com/Henriquemcc/Lista_5_-_Unidade_III_-_Chat_em_Java_-_Redes_de_Computadores_I}
            }{08 2023 -- 09 2023}{}
            Trabalho em grupo desenvolvido na matéria Redes de Computadores I, no qual uma aplicação cliente e servidor deveria ser desenvolvida em Java, utilizando os protocolos (da camada de transporte) TCP e UDP. Para este trabalho, foi desenvolvido um programa de mensagens (chat) no qual dois ou mais clientes, através de um servidor, conseguem trocar mensagens entre si.\\
            \divider
        
            \cvevent{Jogo 8-Puzzle}
            {
            \cvreference{\faGithub}{https://github.com/Henriquemcc/Jogo_-_Inteligencia_Artificial_2023-1}
            \cvreference{|\faGlobe}{https://henriquemcc.github.io/Jogo_-_Inteligencia_Artificial_2023-1/}}{02 2023 -- 06 2023}{}
            Trabalho desenvolvido na matéria Inteligência Artificial, no qual métodos de busca em grafos (A Estrela, Busca Gulosa e Busca Uniforme) deveriam ser utilizados para encontrar uma solução para o Jogo 8-Puzzle.\\
            \divider
            
            \cvevent{Portal de Filmes}
            {
            \cvreference{\faGithub}{https://github.com/Henriquemcc/Portal_de_Filmes_-_Desenvolvimento_de_Interfaces_Web}
            \cvreference{|\faGlobe}{https://henriquemcc.github.io/Portal_de_Filmes_-_Desenvolvimento_de_Interfaces_Web/}}{04 2022 -- 06 2022}{}
            Trabalho prático da matéria Desenvolvimento de Interfaces Web da Pontifícia Universidade Católica de Minas Gerais\\
            \divider

        {\large\color{emphasis}\href{https://www.linkedin.com/in/henriquemcc/details/projects/}{Mais informações no LinkedIn}}
    \end{paracol}
\end{document}
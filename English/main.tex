%%%%%%%%%%%%%%%%%
% This is an sample CV template created using altacv.cls
% (v1.3, 10 May 2020) written by LianTze Lim (liantze@gmail.com). Now compiles with pdfLaTeX, XeLaTeX and LuaLaTeX.
% (v1.6.5b, 27 Jun 2023) forked by Nicolás Omar González Passerino (nicolas.passerino@gmail.com)
%
%% It may be distributed and/or modified under the
%% conditions of the LaTeX Project Public License, either version 1.3
%% of this license or (at your option) any later version.
%% The latest version of this license is in
%%    http://www.latex-project.org/lppl.txt
%% and version 1.3 or later is part of all distributions of LaTeX
%% version 2003/12/01 or later.
%%%%%%%%%%%%%%%%

%% If you need to pass whatever options to xcolor
\PassOptionsToPackage{dvipsnames}{xcolor}

%% If you are using \orcid or academicons
%% icons, make sure you have the academicons
%% option here, and compile with XeLaTeX
%% or LuaLaTeX.
% \documentclass[10pt,a4paper,academicons]{altacv}

%% Use the "normalphoto" option if you want a normal photo instead of cropped to a circle
% \documentclass[10pt,a4paper,normalphoto]{altacv}

%% Fork (before v1.6.5a): CV dark mode toggle enabler to use a inverted color palette.
%% Use the "darkmode" option if you want a color palette used to 
% \documentclass[10pt,a4paper,ragged2e,withhyper,darkmode]{altacv}

\documentclass[10pt,a4paper,ragged2e,withhyper]{altacv}

%% AltaCV uses the fontawesome5 and academicons fonts
%% and packages.
%% See http://texdoc.net/pkg/fontawesome5 and http://texdoc.net/pkg/academicons for full list of symbols. You MUST compile with XeLaTeX or LuaLaTeX if you want to use academicons.

% Change the page layout if you need to
\geometry{left=1.2cm,right=1.2cm,top=1cm,bottom=1cm,columnsep=0.75cm}

% The paracol package lets you typeset columns of text in parallel
\usepackage{paracol}
\usepackage{hyperref}

% Change the font if you want to, depending on whether
% you're using pdflatex or xelatex/lualatex
\ifxetexorluatex
  % If using xelatex or lualatex:
  \setmainfont{Roboto Slab}
  \setsansfont{Lato}
  \renewcommand{\familydefault}{\sfdefault}
\else
  % If using pdflatex:
  \usepackage[rm]{roboto}
  \usepackage[defaultsans]{lato}
  % \usepackage{sourcesanspro}
  \renewcommand{\familydefault}{\sfdefault}
\fi

% Fork (before v1.6.5a): Change the color codes to test your personal variant on any mode
\ifdarkmode%
  \definecolor{PrimaryColor}{HTML}{C69749}
  \definecolor{SecondaryColor}{HTML}{D49B54}
  \definecolor{ThirdColor}{HTML}{1877E8}
  \definecolor{BodyColor}{HTML}{ABABAB}
  \definecolor{EmphasisColor}{HTML}{ABABAB}
  \definecolor{BackgroundColor}{HTML}{191919}
\else%
  \definecolor{PrimaryColor}{HTML}{001F5A}
  \definecolor{SecondaryColor}{HTML}{0039AC}
  \definecolor{ThirdColor}{HTML}{F3890B}
  \definecolor{BodyColor}{HTML}{666666}
  \definecolor{EmphasisColor}{HTML}{2E2E2E}
  \definecolor{BackgroundColor}{HTML}{E2E2E2}
\fi%

\colorlet{name}{PrimaryColor}
\colorlet{tagline}{SecondaryColor}
\colorlet{heading}{PrimaryColor}
\colorlet{headingrule}{ThirdColor}
\colorlet{subheading}{SecondaryColor}
\colorlet{accent}{SecondaryColor}
\colorlet{emphasis}{EmphasisColor}
\colorlet{body}{BodyColor}
\pagecolor{BackgroundColor}

% Change some fonts, if necessary
\renewcommand{\namefont}{\Huge\rmfamily\bfseries}
\renewcommand{\personalinfofont}{\small\bfseries}
\renewcommand{\cvsectionfont}{\LARGE\rmfamily\bfseries}
\renewcommand{\cvsubsectionfont}{\large\bfseries}

% Change the bullets for itemize and rating marker
% for \cvskill if you want to
\renewcommand{\itemmarker}{{\small\textbullet}}
\renewcommand{\ratingmarker}{\faCircle}

%% sample.bib contains your publications
%% \addbibresource{main.bib}

\begin{document}
    \name{Henrique Mendonça Castelar Campos}
    \tagline{Computer Science Student}
    %% You can add multiple photos on the left or right
    \photoL{4cm}{1702063880079}

    \personalinfo{
        \email{henriquemendonacastelar@gmail.com}\smallskip
        \phone{+55 31 9 9506-6143}
        \location{Belo Horizonte, Minas Gerais, Brazil}\\
        \linkedin{henriquemcc}
        \github{Henriquemcc}
        \dockerhub{Henriquemcc}
        % \npm{}
        % \dev{henriquemcc}
        \homepage{henriquemcc.github.io}
        %\medium{}
        %% You MUST add the academicons option to \documentclass, then compile with LuaLaTeX or XeLaTeX, if you want to use \orcid or other academicons commands.
        % \orcid{0000-0002-8726-8936}
        %% You can add your own arbtrary detail with
        %% \printinfo{symbol}{detail}[optional hyperlink prefix]
        % \printinfo{\faPaw}{Hey ho!}[https://example.com/]
        %% Or you can declare your own field with
        %% \NewInfoFiled{fieldname}{symbol}[optional hyperlink prefix] and use it:
        % \NewInfoField{gitlab}{\faGitlab}[https://gitlab.com/]
        \gitlab{henriquemcc}
    }
    
    \makecvheader
    %% Depending on your tastes, you may want to make fonts of itemize environments slightly smaller
    % \AtBeginEnvironment{itemize}{\small}
    
    %% Set the left/right column width ratio to 6:4.
    \columnratio{0.25}

    % Start a 2-column paracol. Both the left and right columns will automatically
    % break across pages if things get too long.
    \begin{paracol}{2}
        % ----- Competências -----
        \cvsection{Technologies I am proficient in}
            \cvtag{Spring Boot}
            \cvtag{Kotlin}
            \cvtag{Python}
            \cvtag{Git}
            \cvtag{GitHub}
            \cvtag{AWS}
            \cvtag{Shell Scripting}
            \cvtag{Linux}
            \cvtag{Java}
            \cvtag{PowerShell}
        
        \divider
        
        \cvsection{Technologies I am familiar with}
            \cvtag{Azure}
            \cvtag{JavaScript}
            \cvtag{HTML5}
            \cvtag{CSS}
            \cvtag{Docker}
            \cvtag{Go}
            \cvtag{C++}
            \cvtag{C\#}
            \cvtag{C}
            \cvtag{\LaTeX}
            \cvtag{Android}\\
            \cvtag{Open MP}
            \cvtag{Nvidia Cuda}

            {\large\color{emphasis}\href{https://www.linkedin.com/in/henriquemcc/details/skills/}{More information on LinkedIn}}
        % ----- Competências -----
        
        % ----- LEARNING -----
        % \cvsection{Learning}
        %     \cvtag{Uno}
        %     \cvtag{Dos}
        %     \cvtag{Tres}
        %     \cvtag{Cuatro}
        %     \cvtag{Cinco}
        %     \cvtag{Seis}
        %     \cvtag{Siete}
        %     \cvtag{Ocho}
        %     \cvtag{Nueve}
        %     \cvtag{Diez}
        %     \medskip
            
        %     \cvtag{Rojo}
        %     \cvtag{Amarillo}
        %     \cvtag{Azul}
        %     \cvtag{Verde}
        %     \cvtag{Violeta}
        %     \cvtag{Naranja}
        %     \cvtag{Marron}
        %     \cvtag{Blanco}
        %     \cvtag{Gris}
        %     \cvtag{Negro}
        % ----- LEARNING -----
        
        % ----- LANGUAGES -----
        % \cvsection{Languages}
        %     \cvlang{Lang 1}{Native}\\
        %     \divider

        %     \cvlang{Lang 2}{Basic / A2}
            %% Yeah I didn't spend too much time making all the
            %% spacing consistent... sorry. Use \smallskip, \medskip,
            %% \bigskip, \vpsace etc to make ajustments.
        % ----- LANGUAGES -----
            
        % ----- REFERENCES -----
        % \cvsection{References}
        %     \cvref{Prof.\ Alpha Beta}{Institute}{a.beta@university.edu}
        %     \divider

        %     \cvref{Boss\ Gamma Delta}{Business}{g.delta@business.com}
        % ----- REFERENCES -----
        
        % ----- MOST PROUD -----
        % \cvsection{Most Proud of}
        
        % \cvachievement{\faTrophy}{Fantastic Achievement}{and some details about it}\\
        % \divider
        % \cvachievement{\faHeartbeat}{Another achievement}{more details about it of course}\\
        % \divider
        % \cvachievement{\faHeartbeat}{Another achievement}{more details about it of course}
        % ----- MOST PROUD -----
        
        % \cvsection{A Day of My Life}
        
        % Adapted from @Jake's answer from http://tex.stackexchange.com/a/82729/226
        % \wheelchart{outer radius}{inner radius}{
        % comma-separated list of value/text width/color/detail}
        % \wheelchart{1.5cm}{0.5cm}{%
        %   6/8em/accent!30/{Sleep,\\beautiful sleep},
        %   3/8em/accent!40/Hopeful novelist by night,
        %   8/8em/accent!60/Daytime job,
        %   2/10em/accent/Sports and relaxation,
        %   5/6em/accent!20/Spending time with family
        % }
        
        % use ONLY \newpage if you want to force a page break for
        % ONLY the current column
        \newpage
        
        %% Switch to the right column. This will now automatically move to the second
        %% page if the content is too long.
        \switchcolumn
        
        % ----- ABOUT ME -----
        \cvsection{Education and Courses}
            \begin{quote}
                I study Computer Science at Pontifical Catholic University of Minas Gerais, currently in the 8th period. Interested in Back-End Development, DevOps, and Cloud Computing.
            \end{quote}
        % ----- ABOUT ME -----
        
        % ----- Experiência -----
        \cvsection{Experience}
            \cvevent{Full stack Developer}{WebTech PUC Minas}{02 2024 --}{Belo Horizonte, Minas Gerais, Brazil}
            Extension work developed at Pontifical Catholic University of Minas Gerais.

            \divider
            \cvevent{Full Stack Developer}{PUCTec}{08 2024 -- 12 2024}{Belo Horizonte, Minas Gerais, Brasil}
            Software development on demand from startups served by PUC Tec's Startup Support Service (SA PUC Tec).
            
            \divider
            \cvevent{Computer Science Peer Tutor}{Pontifical Catholic University of Minas Gerais}{08 2023 -- 12 2023}{Belo Horizonte, Minas Gerais, Brazil}
            Academic tutoring of the discipline Graph Theory and Computability. During academic tutoring, my role consisted of helping students resolve doubts related to the subject.
            
            \divider
            \cvevent{Intern}{Belo Horizonte City Hall}{09 2022 -- 11 2022}{Belo Horizonte, Minas Gerais, Brazil}
            Internal and external service, as help desk, in the SGPREV system.
            
            
            % \divider
        % ----- Experiência -----
        
        % ----- Formação acadêmica -----
        \cvsection{Academic background}
            \cvevent{Bachelor's Degree, Computer Science}{Pontifical Catholic University of Minas Gerais}{02 2018 -- 08 2025}{Belo Horizonte, Minas Gerais, Brazil}
            % \divider
        % ----- Formação acadêmica -----

        % ----- Certificados -----
        \cvsection{Certificates}

        \cvevent{Kotlin and Spring Boot Course - Alura}{\cvreference{\faGlobe}{https://cursos.alura.com.br/degree/certificate/09cbf228-4894-4225-bf0e-6be933822ffe?lang}\cvreference{\faGithub}{https://github.com/Henriquemcc/Forum_-_Formacao_Kotlin_e_Spring_Boot_-_Alura}}{03 2025}{Online}
        \divider
        
        \cvevent{React: Start your full stack project - Alura}{\cvreference{\faGlobe}{https://cursos.alura.com.br/certificate/b4f48602-24d3-4fce-928f-0038f178004a?lang}\cvreference{\faGithub}{https://github.com/Henriquemcc/Alura_Books_-_Formacao_Full_stack_JavaScript_-_Alura}}{01 2025}{Online}
        \divider
        
        \cvevent{Endpoint Security - Cisco}{\cvreference{\faGlobe}{https://www.credly.com/badges/dd945a87-ba32-4732-9ea5-88198208599f/linked_in_profile}}{03 2024}{Online}
        \divider
        
        \cvevent{AWS Certified Cloud Practitioner}{\cvreference{\faGlobe}{https://www.credly.com/badges/7623bc5f-4a7a-49d9-9504-26b399105745/linked_in_profile}}{01 2024}{Belo Horizonte, Minas Gerais, Brazil}
        \divider
        
        \cvevent{Alura's Courses}{\cvreference{\faGlobe}{https://cursos.alura.com.br/user/henriquemcc/fullCertificate/ebc4dcd6245bdf46e4d6ffd89a1e3ec2}}{07 2020 - 12 2023}{Online}
        \divider
        
        \cvevent{B2 First – Score 172 - Cambridge University Press \& Assessment English}{\cvreference{\faGlobe}{https://drive.google.com/file/d/1XlpfYXp5Veeiyn8zAHABk8SSAO36QncZ/view?usp=sharing}}{09 2022}{Belo Horizonte, Minas Gerais, Brazil}
        \divider
        
        \cvevent{Networking Basics - Cisco}{\cvreference{\faGlobe}{https://www.credly.com/badges/c9830260-5298-434e-8955-4eb876480ba6/linked_in_profile}}{09 2022}{Online}
        \divider
        
        \cvevent{Introduction to Cybersecurity - Cisco}{\cvreference{\faGlobe}{https://www.credly.com/badges/4676e79d-3e11-4856-afc7-38b96e1edc95/linked_in_profile}}{08 2022}{Online}
        \divider
        
        \cvevent{Kotlin Language Course - Alura}{\cvreference{\faGlobe}{https://cursos.alura.com.br/degree/certificate/18f608ec-a511-43b4-8586-04c87b079a4c?lang}}{07 2021}{Online}
        \divider
        
        \cvevent{Python Course - Alura}{\cvreference{\faGlobe}{https://cursos.alura.com.br/degree/certificate/b96bda48-dc02-4105-9ca5-ae64c2e135e3?lang}}{08 2020}{Online}
        \divider
        

        {\large\color{emphasis} \href{https://www.linkedin.com/in/henriquemcc/details/certifications/}{More information on LinkedIn}}
        
        % ----- Certificados -----
        
        % ----- Projetos -----
        \cvsection{Projects}

        \cvevent{Prototype of the PUC Tec Demand Registration System}{
            \cvreference{\faGithub}{https://github.com/Henriquemcc/Prototipo_Sistema_Cadastro_Demandas_-_PUC_Tec}
            }{09 2024 - 12 2024}{}
            Prototype of the PUC Tec Demand Registration System Back-End, developed at the PUC Tec Startup Support Service (SA PUC Tec) to meet an internal demand at PUC Tec.\\
            \divider

            \cvevent{Practical Work 1 - Compilers}{
            \cvreference{\faGithub}{https://github.com/Henriquemcc/Trabalho_Pratico_1_-_Compiladores_-_2024-2}
            }{10 2024 - 12 2024}{}
            Practical work for the Compilers discipline of the Computer Science course at the Pontifical Catholic University of Minas Gerais.\\
            \divider

            \cvevent{Bug Watch}{\cvreference{\faGlobe}{https://gitlab.com/henriquemcc/bug_watch}}{08 2024 - 11 2024}{}
            Group work for the discipline Interdisciplinary Work VI in which a distributed, parallel application that uses computer vision should be developed. In this work, a web application for classifying insects through images was developed.\\
            \divider

            \cvevent{Prototype of the Legião dos Corretores System}{
            \cvreference{\faGithub}{https://github.com/Henriquemcc/Prototipo_Sistema_Legiao_dos_Corretores}
            }{10 2024 - 11 2024}{}
            Prototype of the Back-End system in Kotlin Spring Boot, developed at the PUC Tec Startup Support Service (SA PUC Tec) to meet the demand of the startup Legião dos Corretores.\\
            \divider

            \cvevent{Fault Detector with Consensus Simulation - Distributed Computing}{
            \cvreference{\faGithub}{https://github.com/Henriquemcc/Simulacao_Detector_de_Falhas_com_Consenso_-_Computacao_Distribuida_-_2024-2}
            }{11 2024}{}
            Work on the subject Distributed Computing in which a simulation of the functioning of the failure detector with consensus should be developed.\\
            \divider

            \cvevent{Simulation - Distributed Mutual Exclusion - Distributed Computing}{
            \cvreference{\faGithub}{https://github.com/Henriquemcc/Simulacao_-_Exclusao_Mutua_Distribuida_-_Computacao_Distribuida_-_2024-2}
            }{10 2024}{}
            Work on the subject Distributed Computing in which a simulation of the functioning of Distributed Mutual Exclusion algorithms should be developed.\\
            \divider

            \cvevent{Practical Work - Image Processing and Analysis}{
            \cvreference{\faGithub}{https://github.com/Henriquemcc/Trabalho_Pratico_-_Processamento_e_Analise_de_Imagens_-_2024-1}
            }{03 2024 - 06 2024}{}
            Practical work in the Image Processing and Analysis discipline in which a program with a graphical interface should be developed to process and classify, using machine learning techniques, images of medical examination cells.\\
            \divider
            
            \cvevent{Practical Work - Project and Analysis of Algorithms}{
            \cvreference{\faGithub}{https://github.com/Henriquemcc/Trabalho_Pratico_-_Projeto_e_Analise_de_Algoritmos_-_2024-1}
            }{04 2024 - 06 2024}{}
            Practical work in the Project and Analysis of Algorithms discipline in which a program with a graphical interface should be developed to solve problems of optimizing the location of industrial or commercial installations in a given region of interest.\\
            \divider

            \cvevent{Web Server Creation Lab on Microsoft Azure}{
            \cvreference{\faGithub}{https://github.com/WebTech-PUC-Minas/lab-azure-web-server}
            }{04 2024 - 04 2024}{}
            Extension work developed at WebTech PUC Minas in which a laboratory was developed on how to create a simple web server on Microsoft Azure.\\
            \divider
        
            \cvevent{Automatic Deployment Lab with GitHub Actions}{
            \cvreference{\faGithub}{https://github.com/WebTech-PUC-Minas/lab-devops-github-actions}
            }{03 2024 -- 04 2024}{}
            Extension work developed at WebTech PUC Minas in which a laboratory was developed on how to create a workflow in GitHub Actions to automatically integrate a GitHub repository with a web server.\\
            \divider

            \cvevent{Paint}{
            \cvreference{\faGithub}{https://github.com/Henriquemcc/Trabalho_Pratico_-_Computacao_Grafica_-_2024-1}
            }{02 2024 -- 03 2024}{}
            Practical work in the Computer Graphics discipline in which a program with a graphical interface should be developed that used computer graphics algorithms to insert lines, points, circles and polygons, in addition to performing graphical operations of rotation, translation, reflection and scale.\\
            \divider

            \cvevent{Parallel Implementation of Blockchain}
            {
            \cvreference{\faGithub}{https://github.com/Henriquemcc/Projeto_02_-_Implementacao_Paralela_de_Blockchain_-_Computacao_Paralela_-_2023-2}
            }{10 2023 -- 12 2023}{}
            Group work developed in the subject Parallel Computation, in which a Blockchain code should be parallelized using Open MP for Multicore, Open MP for GPU and Nvidia CUDA.\\
            \divider

            \cvevent{Chat in Java}
            {
            \cvreference{\faGithub}{https://github.com/Henriquemcc/Lista_5_-_Unidade_III_-_Chat_em_Java_-_Redes_de_Computadores_I}
            }{08 2023 -- 09 2023}{}
            Group work developed in the subject Computer Networks I, in which a client and server application should be developed in Java, using the protocols (from the transport layer) TCP and UDP. For this work, a messaging program (chat) was developed in which two or more clients, through a server, can exchange messages with each other.\\
            \divider
        
            \cvevent{8-Puzzle Game}
            {
            \cvreference{\faGithub}{https://github.com/Henriquemcc/Jogo_-_Inteligencia_Artificial_2023-1}
            \cvreference{|\faGlobe}{https://henriquemcc.github.io/Jogo_-_Inteligencia_Artificial_2023-1/}}{02 2023 -- 06 2023}{}
            Work developed in the subject Artificial Intelligence, in which graph search methods (A Star, Greedy Search and Uniform Search) should be used to find a solution for the 8-Puzzle Game.\\
            \divider
            
            \cvevent{Movies Portal}
            {
            \cvreference{\faGithub}{https://github.com/Henriquemcc/Portal_de_Filmes_-_Desenvolvimento_de_Interfaces_Web}
            \cvreference{|\faGlobe}{https://henriquemcc.github.io/Portal_de_Filmes_-_Desenvolvimento_de_Interfaces_Web/}}{04 2022 -- 06 2022}{}
            Practical work from the subject Web Interface Development from the Computer Science course at Pontifical Catholic University of Minas Gerais.\\
            \divider

            {\large\color{emphasis}\href{https://www.linkedin.com/in/henriquemcc/details/projects/}{More information on LinkedIn}}
        % ----- Projetos -----
    \end{paracol}
\end{document}